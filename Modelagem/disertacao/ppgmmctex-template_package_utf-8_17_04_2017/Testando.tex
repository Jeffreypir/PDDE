\documentclass[12pt]{beamer}%{{{
\usepackage[portuguese]{babel}
\usepackage{anyfontsize}
\usepackage[all]{xy} 
\usepackage{MnSymbol,extarrows, tikz, pgfplots, graphicx, subfigure,float}
\usepackage{amsfonts,amsmath,amsthm,mathtools,mathrsfs}
\usepackage{ragged2e}
\usetheme{Berkeley}
\logo{\includegraphics{logo.pdf}}
\graphicspath{ {/home/jefferson/Modelamento/} }
\title{Modelo de balan\c co energ\'etico hibr\'ido, contendo fontes intermitentes, baseada em  Programa\c{c}\~{a}o Din\^{a}mica Dual Estoc\'{a}stica}%}}}
\author[]{\textrm{\scriptsize Orientando: Jefferson Bezerra dos Santos\\ Orientadores: Camila de Mara Vital Barros \\
\hspace{2.2cm}S\'ergio de Carvalho Bezerra}}
\institute{Universidade Federal da Para\'iba }
\date{}

\begin{document}
\begin{frame}
\titlepage % Print the title page as the first slide
\end{frame}

\begin{frame}{Sum\'ario}
	\tableofcontents
\end{frame}

\section{Contextualiza\c c\~ao}
\begin{frame}{Contextualiza\c c\~ao}
	\begin{justify}	
	  Com o aumento da demanda por energia el\'etrica e o apelo das fontes renov\'aveis, grandes avan\c cos tecnol\'ogicos
	  s\~ao indispens\'aveis para um crescimento eficiente e sustent\'avel. No Brasil a hidrel\'etrica \'e a principal fonte
	  de gera\c c\~ao de energia. No entanto, devido ao aumento desproporcional da demanda e a escassez de chuvas,
	  tem sido necess\'ario a ativa\c c\~ao de termel\'etricas para suprir a demanda. Consequentemente, isto acarreta em um aumento na fatura
	  dos consumidores residenciais. Vislumbrando a necessidade de realizar um gerenciamento adequado do despacho de energia, de modo a minimizar os custos da gera\c c\~ao e uma diminui\c c\~ao do impacto ambiental, neste trabalho foi proposto um estudo baseado em Programa\c c\~ao
	  Din\^amica Dual Estoc\'astica.
	\end{justify}
\end{frame}

\section{Objetivos}
\begin{frame}{Objetivos}
\subsection{Objetivos Gerais}
Os objetivos gerais deste trabalho s\~ao:
\begin{itemize}
		\justifying
	\item An\'alise da varia\c c\~ao da produtibilidade de um modelo hidrot\'ermico em rela\c c\~ao aos cen\'arios de
		planejamento. 
	\item O modelamento misto de um modelo hidrot\'ermico em regime de complementaridade com fontes de energia renov\'aveis \'eolica e solar.
	\item An\'alise do custo esperado do sistema misto para verifica\c c\~ao de sua viabilidade.
	\item An\'alise dos principais cen\'arios que envolvem o sistema brasileiro. 
	\item Verificar se o modelo misto representa de maneira adequada o problema de planejamento energ\'etico.
\end{itemize}
\end{frame}

\begin{frame}
	\begin{justify}	
	\subsection{Objetivos Espec\'ificos}
	Os objetivos espec\'ificos deste trabalho s\~ao:
	\begin{itemize}
		\justifying
	\item An\'alise da varia\c c\~ao da produtibilidade de um modelo hidrot\'ermico em rela\c c\~ao aos cen\'arios de
		\item Constru\c c\~ao de um algoritmo \'otimo  computacionalmente para o planejamento. 
		\item An\'alise de poss\'iveis erros do modelo.
	\end{itemize}
	\end{justify}
\end{frame}

\section{Metodologia}
\begin{frame}{Metodologia}
	\begin{justify}	
		Primeiramente realizou-se uma revis\~ao bibliogr\'afica sobre o setor energ\'etico brasileiro. Dessa revis\~ao
		observou-se que o Brasil \'e constitu\'ido por um sistema energ\'etico do tipo hidrot\'ermico de grande porte. No qual a
		principal metodologia utilizada \'e t\'ecnica de Programa\c c\~ao Din\^amica Dual Estoc\'astica, pois, tal t\'ecnica
		posssui a capacidade de realizar o planejamento sobre cen\'arios de incerteza. Al\'em, de possui uma implementa\c c\~ao
		computacional relativamente simples.
	\end{justify}
\end{frame}

\begin{frame}
	\begin{justify}	
	A segunda etapa do trabalho foi o entendimento da t\'ecnica e sua reprodu\c c\~ao computacional para teste de verifica\c
	c\~ao se tal t\'ecnica, realmente poderia ser utilizada como as pesquisas afirmavam. Foi verificado que realmente a
	t\'ecnica de Programa\c c\~ao Din\^amica Dual Estoc\'astica possui bons resultados e possui uma flexibilidade no
	planejamento de cen\'arios. 
	\end{justify}
\end{frame}

\begin{frame}
	\begin{justify}	
	Na terceira etapa buscou-se alguma melhoria ou resultado vantajoso do modelo que ainda n\~ao fosse  observado na base
	bibliogr\'afica. Dessa forma, os principais conceitos analizados foram: os cen\'arios, o custo esperado e a
	produtibilidade. Notou-se que uma varia\c c\~ao de produtibilidade fazia que o modelo modifica-se sua configura\c c\~ao
	de forma excepcional. Portanto, foram feitas extensivas simula\c c\~oes para averiguar se tal mudan\c ca realmente
	estava ocorrendo em todos os cen\'arios de planejamento  utilizados. Finalmente, constatou-se que realmente a mudan\c ca
	de produtibilidade fazia que o sistema modifica-se totalmente a sua configura\c c\~ao independentemente da probabilidade
	de ocorr\^encia do cen\'ario.
	\end{justify}
\end{frame}

\section{Despacho de energia}
\begin{frame}{Despacho de energia}
	\begin{justify}	
	No gerenciamento e transmiss\~ao da energia el\'etrica, o Brasil possui o Sistema Interligado Nacional
	(SIN) gerenciado pelo Operador Nacional de Energia (ONS) correspondendo as regi\~oes Sul, Sudeste,
	Centro-Oeste, Nordeste e parte do Norte. O SIN \'e respons\'avel por abrigar cerca de
	$96,6\%$ de toda a capacidade de produ\c c\~ao de energia do Brasil, seja por meio de fontes internas de energia
	ou pela importa\c c\~ao de energia como ocorre na usina de Itaipu mediante o controle compartilhado com o
	Paraguai. A ado\c c\~ao do SIN \'e justificada tendo por base: o interc\^ambio energ\'etico, a
	complementaridade entre fontes de gera\c c\~ao de energia e pela sua capacidade de expans\~ao.
	\end{justify}
\end{frame}

\begin{frame}
	\begin{justify}	
		No planejamento do sistema hidrot\'ermico brasileiro as caracter\'isticas que devem ser consideradas s\~ao:
		\begin{itemize}
		\justifying
		\item \textit{Sazonalidade intra natural}. Al\'em da variabilidade natural ocorre um varia\c c\~ao entre as esta\c c\~oes do ano. 
		\item \textit{A complementariedade e diversidade regional}. As bacias brasileiras possuem caracter\'isticas
			f\'isicas e clim\'a-ticas distintas. 
		\item \textit{O acoplamento espacial}. Na estrutura de cascata as usinas que est\~ao mais perto da jusante possuem depend\^encia
			de usinas mais perto da montante.
		\end{itemize}
	\end{justify}
\end{frame}

\begin{frame}
	\begin{justify}	
		\begin{itemize}
		\justifying
		\item \textit{O acoplamento temporal}. Na estrutura de cascata decis\~oes sobre a utiliza\c c\~ao possuem
			consequ\^encias no futuro. 
		\item \textit{Custo term\'eletrico}. Usinas term\'eletricas possuem um custo alto de produ\c c\~ao el\'etrica em
			rela\c c\~ao as hidrel\'etricas.
		\item \textit{Aspecto ambiental}. Usinas termel\'etricas possuem um alto impacto ambiental ocasionado pela queima de
			combust\'ivel.
	\end{itemize}
	\end{justify}
\end{frame}

\begin{frame}{Usinas em cascata}
	\begin{justify}	
		\begin{figure}[!htpb]
			\centering
			\resizebox{1.0\textwidth}{!}{%
			\input{material/cascataj.tex}}
			\label{st}
		\end{figure}
	\end{justify}
\end{frame}

\begin{frame}{Dilema do operador}
	\begin{justify}	
	 \begin{figure}[!h]
		 \centering
		 \resizebox{1.0\textwidth}{!}{%
		  \xymatrix@=1.0em{
			& & *+[F]{\text{CHUVA}} \ar[r]& *+[F]{\text{DECIS\~AO CORRETA}}\\
			& *+[F]{\text {USAR \ RESERVAT\'ORIO}} \ar[ur] \ar[dr] & &\\
			& & *+[F]{\text{SECA}} \ar[r] & *+[F]{\text{PREJU\'IZO}} \\
			*+ [F]{\text {OPERADOR}} \ar[uur] \ar[ddr] & & & \\
			& & *+[F]{ \text {CHUVA}} \ar[r] & *+ [F]{\text{PREJU\'IZO}}\\
			& *+ [F]{\text {USAR TERMEL\'ETRICA}} \ar[ur] \ar[dr]& &\\
			& & *+[F] {\text {SECA}} \ar[r] & *+[F]{\text{DECIS\~AO CORRETA}}
		 }}
	 \end{figure}
	\end{justify}
\end{frame}

\begin{frame}{Problema de 2 est\'agios}
	\begin{justify}	
	\begin{figure}[!h]
	 \centering
	 \resizebox{1.0\textwidth}{!}{%
	  \xymatrix@=1.0em{
		  & & *+[F]{\text{PLANEJAMENTO}}\ar[d]&\\ 
		  & & *+[F]{\text{CONFIGURA\c C\~AO INICIAL}} \ar[d]&\\  
		  & & *+[F]{\text{1 EST\'AGIO}} \ar[r] \ar[d]& *+[F]{\text{DECIS\~AO VI\'AVEL}} \ar[dd]\\
		  & & *+[F]{\text{CUSTO}} &\\   
		  & & & *+[F]{\text{2 EST\'AGIO}} \ar[d] \\ 
		  & & & *+[F]{\text{CUSTO}} 
	}}
	 \label{estagio}
	\end{figure}
	\end{justify}
\end{frame}

\begin{frame}{Modelo do despacho de energia}
	\begin{justify}	
	\begin{align}
		&\min \langle c_1,x_1\rangle + \langle c_2,x_2\rangle \nonumber \\
		&\mbox{tal que: }	A_1 x_1 \geq b_1 \nonumber \\
		&E_1 x_1 + A_2 x_2 \geq b_2 \nonumber
	\end{align}
	\begin{itemize}
		\justifying
	  \item $c_1$ e $c_2$ s\~ao vetores que representam os custos relacionados ao 1 e 2 est\'agio respectivamente;
	  \item $x_1$ e $x_2$ s\~ao vetores que representam as decis\~oes tomadas 1 e 2 est\'agios respectivamente;
	  \item $b_1$ e $b_2$ s\~ao os vetores de recursos no 1 e 2 est\'agios respectivamente;
	  \item $A_1$ e $A_2$ s\~ao matrizes que representam o acoplamento espacial;
	  \item $E_1$ \'e uma matriz que descreve o acoplamento temporal.
	\end{itemize}
	\end{justify}
\end{frame}

\begin{frame}{Programa\c c\~ao Din\^amica Dual Estoc\'astica}
	\begin{justify}	
		\begin{align*}
		{\text{min}} \ \ \langle c_1,x_1\rangle + p_1 {\beta}_{1} + p_2 {\beta}_{2} \\
		\textrm{Sujeito a:} \quad A_1 x_1 \geq b_1 \\
		{\pi}_{1}^{i}(b_{21} - E_1x_1) - {\beta}_{1} \leq 0 \\ 
		{\pi}_{2}^{j}(b_{22} - E_1x_1) - {\beta}_{2} \leq 0 \\ 
		i = 1, 2, \dots , P \\
		j = 1, 2, \dots , P. 
		\end{align*}
		\begin{itemize}
			\item $p_1$ e $p_2$ probabilidades associadas aos cen\'arios avaliados;
			\item $\beta_1$ e $\beta_2$ escalares derivados da transforma\c c\~ao dual;
			\item ${\pi}_{1}^{i}$ e ${\pi}_{2}^{i}$ v\'ertices do conjunto vi\'avel.
		\end{itemize}
	\end{justify}
\end{frame}

\begin{frame}{Mestre}
	\begin{justify}	
		\begin{align*}
		 \underset {s \backslash a}{\text{min}} \overbrace{\left < c_1,x_1\right >}^{\textrm{Primeiro estágio}} 
		 +\quad \underbrace {p_1 {\beta}_{1} + p_2 {\beta}_{2}}_{\textrm{Segundo estágio}}
		\end{align*}
		\text{Conjunto vi\'avel:}
		$\left \{ 
	 \begin{aligned}
		 A_1 x_1 \geq b_1 \\
		{\pi}_{1}^{i}(b_{21} - e_1x_1) - {\beta}_{1} \leq 0 \\ 
		{\pi}_{2}^{j}(b_{22} - e_1x_1) - {\beta}_{2} \leq 0 \\ 
		i = 1, 2, \dots , p \\
		j = 1, 2, \dots , p. 
	 \end{aligned}
	\right . 
	$
	\end{justify}
\end{frame}

\begin{frame}{Servos}
	\begin{align*}
		\txt{Servo 1:} \hspace{1cm}
	  \begin{split}	
		  \omega_1 = &\max \pi_1 (b_{21} - E_1x_1 ) \\
		&\mbox{tal que: }\pi_1 A_2  \leq c_2.
	  \end{split}
	\end{align*}

		\begin{align*}
			\txt{Servo 2:} \hspace{1cm}
	  \begin{split}	
		  \omega_2 = &\max \pi_2 (b_{22} - E_1x_1 ) \\
		&\mbox{tal que: }\pi_2 A_2  \leq c_2.
	  \end{split}
	\end{align*}
	\end{frame}

	\begin{frame}{Mestre completo}
	 \resizebox{1.2\textwidth}{!}{%
	  \xymatrix@C=0.1em{
		  &&*+[F]{\txt{Mestre completo \\ Resolver o problema encontrando $\beta_1$ e $\beta_2$.\\
		  \\ Verificar converg\^encia para \\ $|\beta_1 -
		  \omega_1|$ e $|\beta_2 - \omega_2| < $ toler\^ancia.}}\ar[dddl] \ar[dddr] 
		  & \hspace{2cm}\\ 
		  & & & & \\
		  & & & & \\
		  &*+[F]{\txt{SERVO 1\\ C\'alculo de $\omega_1$ \\$\omega_1 = \max \pi_1 (b_{21} - E_1x_1)$}}&
		  &*+[F]{\txt{SERVO 2 \\ C\'alculo de $\omega_2$ \\$\omega_2 = \max \pi_2 (b_{22} - E_1x_1 )$}}& & \hspace{2cm}
	}}
\end{frame}

\section{Resultados parciais}
\begin{frame}{Resultados parciais}
O sistema foi configurado de tal forma a garantir a demanda da regi\~ao dada por,
\begin{align*}
{\rho}_1*VTH1 + {\rho}_2*VTH2 + G1 + G2 = DEMANDA,
\end{align*}
onde:
\begin{itemize}
		\justifying
	\item $H_1$ e $H_2$ representam as hidrel\'etricas associadas ao sistema;
	\item $G_1$ e $G_2$ representam as termel\'etricas associadas ao sistema;
	\item $\rho_1$ e $\rho_2$ s\~ao os \'indices de produtibilidade das usinas H1 e H2;
	\item $VTH1$ e $VTH$  os volumes turbinados das hidrel\'etricas associadas.
\end{itemize}
\end{frame}
\begin{frame}{Balan\c co h\'idrico}
O sistema deve preservar o balan\c co hidr\'ico dada por,
{\setlength{\belowdisplayskip}{-4pt}
\begin{align*}
  \displaystyle Vt = VI + VIC - \left( VT + VV \right), 
\end{align*}}
onde : 
\begin{itemize}
		\justifying
	\item $V(t)$ representar o volume em qualquer instante de tempo;
	\item $VI$  volume inicial;
	\item $VIC$ volume incremental;
	\item $VV$ volume vertido.
\end{itemize}
\end{frame}
\begin{frame}{Gera\c c\~ao termel\'etrica}
\begin{justify}	
Por quest\~oes relacionadas ao custo e ao ambiente a gera\c c\~ao das termel\'etricas devem respeitar uma toler\^ancia de
gera\c c\~ao dada por:
\begin{align*}
	G_1 + G_2 \leq G_{max}
\end{align*}
\begin{itemize}
		\justifying
	\item $G_1$ gera\c c\~ao da termel\'etrica 1.
	\item $G_2$ gera\c c\~ao da termel\'etrica 2.
\end{itemize}
\end{justify}	
\end{frame}

\begin{frame}{Produtibilidade $\rho = 1.0$}
\begin{figure}[!ht]
	\centering
		\includegraphics[width=7cm,height=7cm]{prob.0.1/simulacao.1.0/simula.pdf}
	\end{figure}
\end{frame}

\begin{frame}{Produtibilidade $\rho = 1.4$}
\begin{figure}[!ht]
	\centering
		\includegraphics[width=7cm,height=7cm]{prob.0.1/simulacao.1.4/simula.pdf}
\end{figure}
\end{frame}

\begin{frame}{Produtibilidade $\rho = 1.8$}
\begin{figure}[!ht]
	\centering
		\includegraphics[width=7cm,height=7cm]{prob.0.1/simulacao.1.8/simula.pdf}
\end{figure}
\end{frame}

\begin{frame}{Produtibilidade $\rho = 2.0$}
\begin{figure}[!ht]
	\centering
		\includegraphics[width=7cm,height=7cm]{prob.0.1/simulacao.1.8/simula.pdf}
\end{figure}
\end{frame}
\begin{frame}{Configura\c c\~ao para o despacho}
	\begin{figure}[!ht]
		\resizebox{0.8\textwidth}{!}{%
\begin{tikzpicture}
  \draw node[draw, circle] (p10) at (1,1) {P = 1.0};
  \draw node[draw, circle] (p14) at (4,1) {P = 1.4};
  \draw node[draw, circle] (p16) at (7,1) {P = 1.6};
  \draw node[draw, circle] (p18) at (10,1) {P = 1.8};
  \draw node[draw, circle] (p20) at (13,1) {P = 2.0};
  \draw [draw] (p10) to [out=0,in=180] (p14);
  \draw [draw] (p14) to [out=0,in=180] (p16);
  \draw [draw] (p16) to [out=0,in=180] (p18);
  \draw [draw] (p18) to [out=0,in=180] (p20);
  \node [draw](S) at (5.5,5.0) {Ativa\c c\~ao simult\^anea};
  \node [draw, circle, minimum size=5.3cm](c1) at (5.5,5.0) {};
  \node [draw, minimum size=1.0cm] (t1) at (4,4) {T1};
  \node [draw, minimum size=1.0cm] (h2) at (7,4) {H2}; 
  \draw [draw] (t1) to [out=0,in=180] (h2); 
  \draw [draw] (p10) to [out=90,in=180] (t1); 
  \draw [draw] (p18) to [out=90,in=0] (h2); 
  \draw node [] at (5.5, 6.5){Configura\c c\~ao 1};
  \draw node[align=center,draw, minimum size=1.5cm](h) at (11.5,6.4) {Prolongada\\ utiliza\c c\~ao \\de H2};  
  \draw [draw] (p18) to [out=90,in=180] (h);
  \draw [draw] (p20) to [out=90,in=0] (h);
  \draw node[align=center,draw, minimum size=1.5cm](t) at (11.5,4.4) {Posterior\\ ativa\c c\~ao \\de T1};  
  \draw [draw] (h) to [out=270,in=90] (t);
  \node [draw, circle, minimum size=5.2cm](c1) at (11.5,5.2) {};
  \draw node [] at (11.5, 3.2){Configura\c c\~ao 2};
  \node [draw,align=justify, minimum size=2cm]() at (1.0,6.0) {P-Produtibilidade\\
  T1-Termel\'etrica 1\\H2-Hidrel\'etrica 2 };
\end{tikzpicture}}

	\end{figure}
\end{frame}

\section{Proposta da disserta\c c\~ao}
\begin{frame}{Proposta da disserta\c c\~ao}
	\begin{justify}	
		Como abordado o modelo hidrot\'ermico necessita de um alto
		\'indice de produtibilidade associado ao sistema. Esse \'indice n\~ao \'e o ideal pela depend\^encia da produtibilidade
		em rela\c c\~ao a regi\~ao de localiza\c c\~ao da hidrel\'etrica. Desta forma, a proposta da disserta\c c\~ao \'e baseada na
		formula\c c\~ao do planejamento energ\'etico com a utiliza\c c\~ao de fontes de energia auxiliares ao sistema
		hidrot\'ermico, particularmente energias renov\'aveis solar e \'eolica. No intuito do sistema conseguir uma possibilidade
		de adapta\c c\~ao sem a necessidade de altos n\'iveis de produtibilidade. Al\'em, de permitir um planejamento
		sustent\'avel.
	\end{justify}
\end{frame}

\begin{frame}{Energia e\'olica}
	\begin{justify}	
		A energia do tipo \'eolica possui argumentos favor\'aveis a sua utiliza\c c\~ao como: renovabilidade, perenidade, grande
		disponibilidade, independ\^encia de importa\c c\~oes e o custo zero para o suprimento. Para a utiliza\c c\~ao \'eolica o
		Brasil \'e favorecido por uma grande quantidade de suprimento de mat\'eria prima (ventos), sendo caracterizado com uma
		presen\c ca de ventos duas vezes maior que a m\'edia mundial e apresentando uma volatilidade de $5\%$, dessa forma,
		permitindo um melhor controle sobre a previsiblidade  o que auxiliar o planejamento .
	\end{justify}
\end{frame}

\begin{frame}{Energia solar}
	\begin{justify}	
		Com rela\c c\~ao a energia do tipo solar o Brasil possui grande pot\^encial. Uma vez que de acordo com o Plano
		Nacional de 2030 que reproduz os dados do Atlas Solarim\'etico do Brasil, a radia\c c\~ao varia de 8 a 22 MJ
		(Megajoules) por metro quadrado ($m^{2}$) durante o dia, tendo como as menores varia\c c\~oes os meses de maio e de
		julho com cerca de 8 a 18 MJ (Megajoules) por metro quadrado ($m^{2}$).  Outro aspecto de relev\^ancia
		mencionado no estudo deve-se a regi\~ao nordeste  dispor de um n\'ivel de radia\c c\~ao compar\'avel as melhores
		regi\~oes do mundo.
	\end{justify}
\end{frame}
\end{document}


