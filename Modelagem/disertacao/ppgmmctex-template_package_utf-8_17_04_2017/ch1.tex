\chapter{Introdução}
Neste cap\'itulo \'e abordado a contextualiza\c c\~ao do problema de planejamento energ\'etico. Os objetivos da
 pesquisa, a metodologia utilizada e por fim a estrutura do trabalho. 
\section{Contextualiza\c c\~ao}
A crescente necessidade pelo atendimento \`a demanda tem ocasionado um aumento da complexidade dos
sistemas de gera\c c\~ao de energia el\'etrica. Em contrapartida, a pesquisa por uma gera\c c\~ao de energia que favore\c
ca o desenvolvimento sustent\'avel tornou-se um dos principais temas debatidos no cen\'ario internacional.
O sistema brasileiro \'e constitu\'ido predominantemente por um sistema interligado
hidrot\'ermico, tendo como caracter\'isticas principais o interc\^ambio de energia entre regi\~oes e a possibilidade de
complementaridade existente entre as hidrel\'etricas e as termel\'etricas \cite{tom} e \cite{an}. 

Em um planejamento hidrot\'ermico os aspectos
de relev\^ancia s\~ao: acoplamento espacial, acoplamento temporal e o componente estoc\'astico dos reservat\'orios. Em cada
est\'agio do planejamento \'e necess\'aria a tomada de decis\~ao fazendo-se a escolha pela quantidade gerada de energia
proveniente das termel\'etricas e das hidrel\'etricas. Neste
contexto, atualmente destaca-se a Programa\c c\~ao Din\^amica Dual\abbrev{PDDE}{Programa\c c\~ao Din\^amica Dual
Estoc\'astica}
Estoc\'astica (PDDE), pois possibilita uma flexibilidade para a descri\c c\~ao do acoplamento temporal e espacial existente entre as
usinas hidrel\'etricas, al\'em de permitir o  planejamento em v\'arios cen\'arios de aflu\^encias proporcionando-se a
modelagem da incerteza dos reservat\'orios. Contudo, dependendo da quantidade elevada de cen\'arios  para o planejamento
nem sempre \'e poss\'ivel que a t\'ecnica de PDDE obtenha uma configura\c c\~ao \'otima para todos os cen\'arios
considerados, tornando-se sua principal desvantagem.

Diversos trabalhos na literatura utilizam a PDDE como forma de planejamento. Um estudo recente da t\'ecnica
de constru\c c\~ao de \'arvore de cen\'arios para a PDDE pode ser encontrado em \cite{Reben}.
O estudo sobre modelagem hidrot\'ermica n\~ao convexa utilizando PDDE para restri\c c\~oes n\~ao lineares envolvendo reservat\'orios de
hidrel\'etricas \'e encontrado em \cite{torres}.
Uma an\'alise comparativa entre Programa\c c\~ao Din\^amica Primal Estoc\'astica e PDDE para modelo
hidrot\'ermico de longo prazo \'e descrita em \cite{soares}. A utiliza\c c\~ao de um m\'etodo decisor para aux\'ilio da
PDDE foi proposto por \cite{vitor}. Um m\'etodo baseado em amostragem foi utilizado para constru\c c\~ao de cen\'arios
por \cite{homem}. Por fim, foi proposto por \cite{dematos} uma metodologia para o melhoramento da PDDE em rela\c c\~ao
ao aspecto computacional por meio da sele\c c\~ao de hiperplanos utilizados nas itera\c c\~oes. 

Neste trabalho foi feita a an\'alise dos pontos cr\'iticos de demanda e de produtibilidade em modelos hidrot\'ermicos que utilizam a PDDE.
Tendo como intuito verificar quais as consequ\^encias nas varia\c c\~oes da demanda e da produtibilidade no valor total do custo esperado.
Vale destacar a obten\c c\~ao de duas configura\c c\~oes de despacho considerando-se o aspecto macro, ou seja, ordem de acionamento das 
hidrel\'etricas e das termel\'etricas, com o aumento da demanda e dos n\'iveis de produtibilidade.
Ao final se detecta qual cen\'ario apresenta o menor custo esperado de produ\c c\~ao de gera\c c\~ao de energia el\'etrica,
dentro das situa\c c\~oes modeladas.

 \section{Objetivos}
\subsection{Objetivos Gerais}
Os objetivos gerais deste trabalho s\~ao:
\begin{itemize}
	\item An\'alise da varia\c c\~ao da produtibilidade de um modelo hidrot\'ermico em rela\c c\~ao aos cen\'arios de
		planejamento. 
	\item An\'alise do custo esperado do sistema misto para verifica\c c\~ao de sua viabilidade.
	\item An\'alise dos principais cen\'arios que envolvem o sistema brasileiro. 
	\item Verificar se o modelo misto representa de maneira adequada o problema de planejamento energ\'etico.
\end{itemize}

\subsection{Objetivos Espec\'ificos}
Os objetivos espec\'ificos deste trabalho s\~ao:
\begin{itemize}
	\item Constru\c c\~ao de um algoritmo \'otimo  computacionalmente para o planejamento. 
	\item An\'alise de poss\'iveis erros do modelo.
\end{itemize}
\section{Metodologia}
Primeiramente realizou-se uma revis\~ao bibliogr\'afica sobre o setor energ\'etico brasileiro. Dessa revis\~ao
observou-se que o Brasil \'e constitu\'ido por um sistema energ\'etico do tipo hidrot\'ermico de grande porte. No qual a
principal metodologia utilizada \'e t\'ecnica de Programa\c c\~ao Din\^amica Dual Estoc\'astica, pois, tal t\'ecnica
posssui a capacidade de realizar o planejamento sobre cen\'arios de incerteza. Al\'em, de possui uma implementa\c c\~ao
computacional relativamente simples.

A segunda etapa do trabalho foi o entendimento da t\'ecnica e sua reprodu\c c\~ao computacional para teste de verifica\c
c\~ao se tal t\'ecnica, realmente poderia ser utilizada como as pesquisas afirmavam. Foi verificado que realmente a
t\'ecnica de Programa\c c\~ao Din\^amica Dual Estoc\'astica possui bons resultados e  flexibilidade no
planejamento de cen\'arios. 

Na terceira etapa foram observadas melhorias ou vantagens do modelo que ainda n\~ao foram  observadas na base
bibliogr\'afica. Dessa forma, os principais conceitos analizados foram: os cen\'arios, o custo esperado e a
produtibilidade.
\section{Estrutura da proposta}
O presente trabalho \'e estruturado da seguinte forma:

No \textbf{capt\'itulo 1} \'e apresentado o contexto do problema, os objetivos gerais e espec\'ificos do trabalho, 
a metodologia e a estrutura do trabalho.

No \textbf{cap\'itulo 2} \'e decrito o estado da arte com as atuais e principais pesquisas relevantes para o trabalho.

No \textbf{cap\'itulo 3} \'e tratado o problema para o despacho hidrot\'ermico e a estrutura do modelo de planejamento
		baseado em Programa\c c\~ao Din\^amica Dual Estoc\'astica.

No \textbf{cap\'itulo 4} \'e apresentado os resultados parciais do modelo de despacho para o caso hidrot\'ermico. Em particular a an\'alise da varia\c c\~ao de produtibilidade para o modelo.

No \textbf{cap\'itulo 5} \'e abordado a proposta da disserta\c c\~ao, isto \'e,  a utiliza\c c\~ao de fontes de energia
renov\'aveis em regime de complementaridade com as hidrel\'etricas e termel\'etricas.

