\chapter{Proposta da Disserta\c c\~ao}
Nos cap\'itulos anteriores foram abordados a problem\'atica do planejameto hidrot\'ermico com a utiliza\c c\~ao da
t\'ecnica de Programa\c c\~ao Din\^amica Dual Estoc\'astica. Pelos resultados parciais da pesquisa nota-se que para um
desempenho satisfat\'orio do sistema considerando-se aspectos como: os picos de demanda, o custo associado a gera\c
c\~ao de energia el\'etrica e a quest\~ao ambiental. O modelo hidrot\'ermico necessita de um alto
\'indice de produtibilidade associado ao sistema. Esse \'indice n\~ao \'e o ideal pela depend\^encia da produtibilidade
em rela\c c\~ao a regi\~ao de localiza\c c\~ao da hidrel\'etrica. Desta forma, a proposta da disserta\c c\~ao \'e baseada na
formula\c c\~ao do planejamento energ\'etico com a utiliza\c c\~ao de fontes de energia auxiliares ao sistema
hidrot\'ermico, particularmente energias renov\'aveis solar e \'eolica. No intuito do sistema conseguir uma possibilidade
de adapta\c c\~ao sem a necessidade de altos n\'iveis de produtibilidade. Al\'em, de permitir um planejamento
sustent\'avel.

A escolha pelas fontes mencionadas se baseia em qualidades favor\'aveis a matriz energ\'etica brasileira. Primeiramente,
a energia do tipo \'eolica possui argumentos favor\'aveis a sua utiliza\c c\~ao como: renovabilidade, perenidade, grande
disponibilidade, independ\^encia de importa\c c\~oes e o custo zero para o suprimento. Para a utiliza\c c\~ao \'eolica o
Brasil \'e favorecido por uma grande quantidade de suprimento de mat\'eria prima (ventos), sendo caracterizado com uma
presen\c ca de ventos duas vezes maior que a m\'edia mundial e apresentando uma volatilidade de $5\%$, dessa forma,
permitindo um melhor controle sobre a previsiblidade  o que auxiliar o planejamento \cite{an}. 

Com rela\c c\~ao a energia do tipo solar o Brasil possui grande pot\^encial. Uma vez que de acordo com o Plano
Nacional de 2030 que reproduz os dados do Atlas Solarim\'etico do Brasil, a radia\c c\~ao varia de 8 a 22 MJ
(Megajoules) por metro quadrado ($m^{2}$) durante o dia, tendo como as menores varia\c c\~oes os meses de maio e de
julho com cerca de 8 a 18 MJ (Megajoules) por metro quadrado ($m^{2}$) \cite{an}.  Outro aspecto de relev\^ancia
mencionado no estudo deve-se a regi\~ao nordeste  dispor de um n\'ivel de radia\c c\~ao compar\'avel as melhores
regi\~oes do mundo.

