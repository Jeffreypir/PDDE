
\begin{abstract}
  Com o aumento da demanda por energia el\'etrica e o apelo das fontes renov\'aveis, grandes avan\c cos tecnol\'ogicos
  s\~ao indispens\'aveis para um crescimento eficiente e sustent\'avel. No Brasil a hidrel\'etrica \'e a principal fonte
  de gera\c c\~ao de energia. No entanto, devido ao aumento desproporcional da demanda e a escassez de chuvas,
  tem sido necess\'ario a ativa\c c\~ao de termel\'etricas para suprir a demanda. Consequentemente, isto acarreta em um aumento na fatura
  dos consumidores residenciais. Neste contexto, este trabalho tem como finalidade analisar,
  as consequ\^encias que o aumento da demanda e varia\c c\~oes de produtibilidade 
  associados a problemas energ\'eticos, podem ocasionar no gerenciamento do balan\c co energ\'etico.
  Vislumbrando a necessidade de realizar um gerenciamento adequado do despacho de energia, de modo a minimizar os custos da gera\c
  c\~ao e uma diminui\c c\~ao do impacto ambiental, neste trabalho foi proposto um estudo baseado em Programa\c c\~ao
  Din\^amica Dual Estoc\'astica para sistemas hidrot\'ermicos (hidrel\'etricas e termel\'etricas).
  De fato, ao variar a produtibilidade e a demanda se identifica qual a configura\c c\~ao de despacho que apresenta 
  o menor custo esperado para a gera\c c\~ao do sistema. 
\end{abstract}

%\begin{abstract}
%  Com o aumento da demanda por energia el\'etrica e o apelo das fontes renov\'aveis, grandes avan\c cos tecnol\'ogicos
%  s\~ao indispens\'aveis para um crescimento eficiente e sustent\'avel. No Brasil a hidrel\'etrica \'e a principal fonte de gera\c c\~ao de energia. No entanto, devido ao aumento desproporcional da demanda e a escassez de chuvas, tem sido necess\'ario a ativa\c c\~ao de termel\'etricas para suprir a demanda. Consequentemente, isto acarreta em um aumento na fatura
%  dos consumidores residenciais e dependendo da gravidade da situa\c c\~ao medidas de racionamento podem serem
%  implementadas.
%  Neste contexto, este trabalho tem como objetivo propor o modelo hidrot\'ermico misto com as fontes de energia
%  renov\'aveis solar e e\'olica. O modelo \'e baseado na t\'ecnica de programa\c c\~ao din\^amica dual estoc\'astica.
%  Uma vez que a t\'ecnica utilizada permite a modelagem flex\'ivel dos componentes estoc\'asticos associados ao sistema
%  energ\'etico.
%  \end{abstract}

