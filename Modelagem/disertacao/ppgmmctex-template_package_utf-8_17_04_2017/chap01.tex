\chapter{Introdução}
\section{Contextualiza\c c\~ao}
A crescente necessidade pelo atendimento \`a demanda tem ocasionado um aumento da complexidade dos
sistemas de gera\c c\~ao de energia el\'etrica. Em contrapartida, a pesquisa por uma gera\c c\~ao de energia que favore\c
ca o desenvolvimento sustent\'avel tornou-se um dos principais temas debatidos no cen\'ario internacional. O sistema brasileiro \'e constitu\'ido predominantemente por um sistema interligado
hidrot\'ermico, tendo como caracter\'isticas principais o interc\^ambio de energia entre regi\~oes e a possibilidade de
complementaridade existente entre as hidrel\'etricas e as termel\'etricas \cite{tom}. 

Em um planejamento hidrot\'ermico os aspectos
de relev\^ancia s\~ao: acoplamento espacial, acoplamento temporal e o componente estoc\'astico dos reservat\'orios. Em cada
est\'agio do planejamento \'e necess\'aria a tomada de decis\~ao fazendo-se a escolha pela quantidade gerada de energia
proveniente das termel\'etricas e das hidrel\'etricas. Neste
contexto, atualmente destaca-se a Programa\c c\~ao Din\^amica Dual
Estoc\'astica (PDDE), pois possibilita uma flexibilidade para a descri\c c\~ao do acoplamento temporal e espacial existente entre as
usinas hidrel\'etricas, al\'em de permitir o  planejamento em v\'arios cen\'arios de aflu\^encias proporcionando-se a
modelagem da incerteza dos reservat\'orios\cite{an}. Contudo, com a crescente preocupa\c c\~ao ambiental e a descoberta de
novas  fontes de energia renov\'aveis \'e necess\'ario a adequa\c c\~ao do modelo hidrot\'ermico brasileiro aos novos fatores para
o planejamento energ\'etico no intuito de viabilizar um fornecimento de energia el\'etrica de baixo custo e que seja
suficiente ao atendimento da demanda dos v\'arios setores brasileiros. As fontes de energia que na atualidade se
destacam s\~ao: a energia \'eolica e a solar.  

O Brasil possui em rela\c c\~ao a energia \'eolica  caracter\'isticas favor\'aveis a sua utiliza\c c\~ao apresentado em
termos de ventos uma quantidade duas vezes maior que \`a m\'edia mundial. Dessa forma, permitindo maior previsibilidade
do volume produzido de energia. Al\'em, da velocidade dos ventos ser maior em per\'iodos de estiagem. Portanto,
permitindo a utiliza\c c\~ao em regime de complementaridade com as usinas hidrel\'etricas \cite{an}. Em rela\c c\~ao a
energia solar o Brasil possui caracter\'isticas semelhantes em rela\c c\~ao a \'eolica por apresentar altos n\'iveis de
radia\c c\~ao solar \cite{an}. Os principais m\'etodos utilizados para a obten\c c\~ao de energia solar s\~ao: o
heliot\'ermico e o fotovoltaico. O heliot\'ermico \'e a obten\c c\~ao da energia por um por um processo indireto no qual
a irradia\c c\~ao \'e convertida em calor e utilizada para gera\c c\~ao de energia por meio das termel\'etricas. Sua
utiliza\c c\~ao necessita de alta intensidade solar caracter\'isticas de regi\~oes que possuem poucas nuvens e
ba\'ixos \'indices pluviom\'etricas, por exemplo, o semi-\'arido brasileiro\cite{an}.

Nesse contexto, o presente trabalho explorar a utiliza\c c\~ao de uma modelagem mista com a utiliza\c c\~ao de fontes de
energia renov\'aveis solar e e\'olica em regime de complementaridade com o modelo hidrot\'ermico brasileiro
utilizando-se da t\'ecnica de PDDE. Uma vez que PDDE permite uma modelagem flex\'ivel de problemas que possuem
componentes estoc\'asticos. Al\'em de ser amplamente utilizada no cen\'ario de modelagem de problemas energ\'eticos. Em
particular, tem-se o intuito de desenvolver um algoritmo capaz de descrever de maneira realista os cen\'arios que o
sistema brasileiro pode enfrentar. Dessa
forma, auxiliando o planejamento energ\'etico e permitindo que o sistema energ\'ertico brasileiro tenha condi\c c\~oes de
adapta\c c\~ao ao ambiente. Contudo, na perspectiva de uma gera\c c\~ao  limpa e de ba\'ixo custo.
\section{Objetivos}
\subsection{Objetivos Gerais}
Os objetivos gerais deste trabalho s\~ao:
\begin{itemize}
	\item O modelamento misto de um modelo hidrot\'ermico em regime de complementaridade com fontes de energia renov\'aveis \'eolica e solar.
	\item An\'alise do custo esperado do sistema misto para verifica\c c\~ao de sua viabilidade.
	\item An\'alise dos principais cen\'arios que envolvem o sistema brasileiro. 
	\item Verifica\c c\~ao se o modelo misto representa de maneira adequada o problema de planejamento energ\'etico.
\end{itemize}

\subsection{Objetivos Espec\'ificos}
Os objetivos espec\'ificos deste trabalho s\~ao:
\begin{itemize}
	\item Constru\c c\~ao de um algoritmo \'otimo  computacionalmente para o planejamento. 
	\item An\'alise de poss\'iveis erros do modelo.
\end{itemize}
\section{Metodologia}
Primeiramente realizou-se uma revis\~ao bibliogr\'afica sobre o setor energ\'etico brasileiro. Dessa revis\~ao
observou-se que o Brasil \'e constitu\'ido por um sistema energ\'etico do tipo hidrot\'ermico de grande porte. No qual a
principal metodologia utilizada \'e t\'ecnica de Programa\c c\~ao Din\^amica Dual Estoc\'astica, pois, tal t\'ecnica
posssui a capacidade de realizar o planejamento sobre cen\'arios de incerteza. Al\'em, de possui uma implementa\c c\~ao
computacional relativamente simples.

A segunda etapa do trabalho foi o entendimento da t\'ecnica e sua reprodu\c c\~a computacional para teste de verifica\c
c\c c\~ao se tal t\'ecnica realmente poderia ser utilizada como as pesquisas afirmavam. Foi verificado que realmente a
t\'ecnica de Progrma\c c\~ao din\^amica dual estoc\'astica possui bons resultados e possui uma flexibilidade no
planejamento de cen\'arios. 

Na terceira etapa buscou-se alguma melhoria ou resultado vantajoso do modelo que ainda n\~ao fosse  observado na base
bibliogr\'afica. Dessa forma, os principais conceitos analizados foram: os cen\'arios, o custo esperado e a
produtibilidade. Notou-se que uma varia\c c\~ao de produtibilidade fazia que o modelo modifica-se sua configura\c c\~ao
de forma excepcional. Portanto, foram feitas extensivas simula\c c\~oes para averiguar se tal mudan\c ca realmente
estava ocorrendo em todos os cen\'arios de planejamento  utilizados. Finalmente constatou-se que realmente a mudan\c ca
de produtibilidade fazia que o sistema modifica-se totalmente a sua configura\c c\~ao independentemente a probabilidade
de ocorr\^encia do cen\'ario.

\section{Estrutura da proposta}
O presente trabalho \'e estruturado da seguinte forma:
\begin{description}
	\item[Cap\'itulo 1: Introdu\c c\~ao.]
		Nesse cap\'itulo \'e apresentado o contexto do problema, os objetivos gerais e espec\'ificos do trabalho, a
		metodologia e a estrutura do trabalho.
	\item[Cap\'itulo 2: Estado da arte.]
		Descrever as atuais e principais pesquisas relevantes para o trabalho.
	\item[Cap\'itulo 3: Despacho de Energia.]
		Nesse cap\'itulo \'e tratado o sistema de energia brasileiro, a diferencia\c c\~ao entre matriz energ\'etica e
		matriz el\'etrica. Al\'em, da constru\c c\~ao  da defini\c c\~ao do despacho de energia, exemplifica\c c\~ao das
		caracter\'isticas do sistema hidrot\'ermico, solar e \'eolico.
	\item[Cap\'itulo 4: Otimiza\c c\~ao.]
		Desenvolvimento da teoria de otimiza\c c\~ao necess\'aria para o modelo de despacho.
	\item[Cap\'itulo 5: Probabilidade.]
		Desenvolvimento da teoria de probabilidade necess\'aria para o modelo de despacho.
	\item[Cap\'itulo 6: Programa\c c\~ao Din\^amica Dual Estoc\'astica.]
		Apresenta\c c\~ao e descri\c c\~ao do modelo a ser utilizado.
	\item[Cap\'itulo 7: Resultados Parciais.]
		Os resultados parciais do modelo de despacho para o caso hidrot\'ermico. Em particular a an\'alise da varia\c
		c\~ao de produtibilidade para o modelo.
	\item[Cap\'itulo 8: Proposta de Disserta\c c\~ao.]
		trabalhando nisso
\end{description}

