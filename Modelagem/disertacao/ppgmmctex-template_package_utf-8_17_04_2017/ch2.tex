\chapter{Estado da Arte}
No cap\'itulo anterior foi abordado a contextualiza\c c\~ao do trabalho, os objetivos, a metodologia da pesquisa e a
estrutura do trabalho. Neste cap\'itulo \'e realizada uma revis\~ao bibliogr\'afica sobre o estado da arte para o tema
estudado.
Verificou-se que no planejamento energ\'etico para sistemas hidrot\'ermicos a principal t\'ecnica utilizada \'e a
Programa\c c\~ao Din\^amica Dual Estoc\'astica. Desta forma, foi realizado um profundo estudo sobre a t\'ecnica desde sua
fundamenta\c c\~ao te\'orica at\'e sua implementa\c c\~ao.
\section{Revis\~ao bibliogr\'afica}
A t\'ecnica da Programa\c c\~ao Din\^amica Dual Estoc\'astica(PDDE) \'e baseda na decomposi\c c\~ao de Benders \cite{benders}.
Sua ideia principal \'e evitar o problema relacionado com a discretiza\c c\~ao do espa\c co de estados por meio da utiliza\c c\~ao da
dualidade de um problema de otimiza\c c\~ao \cite{alexey} e \cite{boyd}.

Em 2004 foi realizado uma an\'alise da Programa\c c\~ao  Din\^amica Dual Estoc\'astica por \cite{soares} para malha
aberta parcial em compara\c c\~ao com a malha fechada do sistema. Sua principal justificativa deve-se ao fato que a PDDE
mesmo em sistemas simples demanda um esfor\c co computacional n\~ao desprez\'ivel. A sua principal contribui\c c\~ao
foi a
modelagem por meio de um modelo auxiliar determin\'istico o qual tem como fun\c c\~ao a tomada de decis\~ao para a
mudan\c ca de cen\'arios no modelo de PDDE. Foi observado resultados vantajosos para sua metodologia, particularmente em
per\'iodos de estriagem.

O m\'etodo baseado em constru\c c\~oes de cen\'arios por meio de amostragem foi proposto por \cite{homem} utilizando-se
de simula\c c\~ao de Monte Carlo usual, hipercubo latino e Monte Carlo Randonizado. Os dois objetivos principais na
utiliza\c c\~ao das t\'ecnicas mencionadas foram a constru\c c\~ao dos cen\'arios. Al\'em, de utilizar t\'ecnicas
baseadas em testes de hip\'otese para a constru\c c\~ao de crit\'erios de parada robustos para o problema de
planejamento energ\'etico por meio da PDDE. 

Na utiliza\c c\~ao da t\'ecnica da PDDE o problema relacionado aos cortes por hiperplanos para v\'arios cen\'arios
pode ser necess\'ario um grande esfor\c co computacional. Nessa perspectiva foi proposto por \cite{dematos} melhorias da
t\'ecnica de PDDE em rela\c c\~ao a implementa\c c\~ao computacional. O metodologia utilizada foi a
utiliza\c c\~ao de uma medida para o selecionamento dos cortes por hiperplanos para cada itera\c c\~ao. O principal
resultado foi diminui\c c\~ao do tempo computacional do modelo sem percas siginificativas na qualidade da solu\c c\~ao. 

No planejamento  hidrot\'ermico as incertezas do ambiente possuem grande import\^ancia. Em 2012 foi proposto por
\cite{torres}  a
modelagem do planejamento hidrot\'ermico por meio da PDDE para um conjunto n\~ao convexo. O seu intuito era o
planejamento
utilizando o modelo hidrot\'ermico para o caso do fen\^omeno de cabe\c ca d'\'agua de uma usina hidrel\'etrica. Pelo fen\^omeno ser
de natureza n\~ao linear houve uma modifica\c c\~ao no modelo de PDDE para a adequa\c c\~ao do planejamento. Uma vez que
a
modelagem para o caso que o  conjunto vi\'avel \'e de natureza n\~ao convexa n\~ao \'e poss\'ivel uma a aplica\c c\~ao direta do m\'etodo
de relaxa\c c\~ao de Benders. Sua principal contribui\c c\~ao foi superar a dificuldade apresentada utilizando-se da
t\'ecnica de RL para o planejamento estoc\'astico n\~ao linear. 

O m\'etodo de decis\~ao do sistema possui caracter\'istica importante para o problema do tipo PDDE. Nesse intuito foi
formulado em 2012 \cite{vitor} um m\'etodo para auxiliar a t\'ecnica de PDDE no processo decisivo. O m\'etodo
utiliza a fun\c c\~ao denominada fun\c c\~ao de utilidade com a qual \'e poss\'ivel a an\'alise de decisores para o
planejamento permitindo obter informa\c c\~oes sobre os riscos no planejamento. A metodologia utilizada nessa abordagem
foi constitu\'ida por tr\^es formas para o planejamento. A primeira baseada em uma discretiza\c c\~ao, a segunda baseada
em uma forma cont\'inua e a terceira utilizando-se um modelo j\'a desenvolvido como forma de apoio.

Na utiliza\c c\~ao da PDDE existe duas principais abordagens a constru\c c\~ao de cen\'arios e a utiliza\c
c\~ao de m\'etodos de amostragem. Em 2016 foi proposto por \cite{Reben} a unifica\c c\~ao das t\'ecnicas no intuito da
formula\c c\~ao de um modelo robusto de PDDE. Pois, os m\'etodos por cen\'arios possuem a desvantagem da necessidade de
um grande n\'umero de cen\'arios. Por outro lado m\'etodos de amostragem necessitam de uma escolha razo\'avel para a
descri\c c\~ao do conjunto observado podendo ser uma representa\c c\~ao razo\'avel ou  n\~ao representa de maneira
adequada. Entre os principais resultados da abordagem foram: a contru\c c\~ao de algoritmo robusto para o modelamento da
incerteza por meio de cen\'arios e amostragem, particularmente o modelo foi aplicado para um caso real do Panam\'a.

\section{Considera\c c\~oes finais}
Nesse cap\'itulo foram abordados alguns dos trabalhos sobre o tema proposto. Vale destacar que o planejamento energ\'etico possui
caracter\'isticas que o tornam bastante complexo. Desta forma, h\'a uma grande quantidade de metodologias e t\'ecnicas na
busca de um planejamento eficiente, contudo, a t\'ecnica que pode ser destacada \'e a PDDE.  
