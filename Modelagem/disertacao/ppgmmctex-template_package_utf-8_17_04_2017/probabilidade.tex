No cap\'itulo anterior estabeleu-se os conceitos da teoria da otimiza\c c\~ao relevantes para essa pesquisa em
particular problemas de programa\c c\~ao linear. Contudo, para a formula\c{c}\~{a}o de um modelo de dispacho
de energia considerando somente o aspecto deterministico, isto \'e, um problema de programa\c c\~ao linear
esse n\~ao possui as caracter\'isticas necess\'arias para a an\'alise do sistema hidrot\'ermico, pois, as
mudan\c{c}as que podem ocorrer no ambiente acarretam um s\'{e}rio impacto no sistema. Portanto, para modelos
hidrot\'{e}rmicos \'{e} necess\'{a}rio o conhecimento sobre a teoria das probabilidades. Este cap\'itulo
aborda os principais conceitos para o entendimento do modelo de dispacho de energia considerando a variedade
de cen\'{a}rios que podem ocorrer. Para um estudo mais rigoroso sobre a teoria da probabilidade consulte
\cite{james} e \cite{magalhaes}.  \par Na natureza encontramos uma s\'{e}rie de situa\c{c}\~{o}es que envolvem
algum tipo de incertezas denominados fen\^{o}menos ou experimentos aleat\'{o}rios. Por exemplo, o
lan\c{c}amento de um dado, a previs\~{a}o do clima e o lan\c camento de um moeda s\~ao exemplos t\'ipicos.  O
espa\c{c}o amostral \'{e} o conjunto de todos os resultados poss\'{i}veis sendo representado  por $\Omega$
podendo ser enumer\'{a}vel, finito ou infinito \cite{magalhaes}. Neste cap\'itulo ser\'a adotado as seguinte
conven\c c\~oes: 
	\begin{itemize} \item $\omega$ \'e um elemento do espa\c co amostral $\Omega$;
		\item A utiliza\c c\~ao de letras mai\'usculas para a nomea\c c\~ao de subconjuntos, por exemplo $A \subset 
		\Omega$; 
		\item  O conjunto vazio ser\'a denotado por $\emptyset$.
	\end{itemize} Um exemplo simples da
utiliza\c c\~ao do espa\c co amostral $\Omega$ \'e o lan\c camento de um dado. Seja o lan\c{c}amento de um
dado desejando-se uma descri\c c\~ao do espa\c co amostral desse evento deve-se elencar todos os resultados
poss\'iveis, ou seja, $ 1, 2, 3, 4, 5, 6$. Portanto, o espa\c{c}o amostral ser\'{a} dado por, $$\Omega =
\{1,2,3,4,5,6\}.$$ Alguns dos subconjuntos de $\Omega$ podem ser elencados da seguinte forma, 
	\begin{align*} 
		A = \{1,2,3\}, B = \{6\}.
	\end{align*}
A representa\c{c}\~{a}o pela linguagem dos conjuntos \'e usualmente
empregada em probabilidade com o intuito de uma descri\c c\~ao  objetiva dos eventos. Por exemplo,
considerando-se a descri\c c\~ao do evento ``sair um n\'{u}mero menor que 4'' na linguagem dos conjuntos a
representa\c{c}\~{a}o desse evento poderia ser, $$ C = \{\omega \in \Omega; \omega < 4 \}$$ desta forma, o
evento de interesse est\'{a} bem formulado. Na modelagem de um evento probabil\'istico por meio da linguagem
dos conjuntos deve-se tomar alguns cuidados b\'asicos, por exemplo, 
	\begin{align*} 
		6 \in \Omega, 6 \subset \Omega.
	\end{align*}
Nota-se que a segunda afirma\c c\~ao n\~ao possue sentido algum.  As principais
nota\c{c}\~{o}es para as opera\c{c}\~{o}es entre conjuntos s\~{a}o:
	\begin{description} 
		\item (i) $A^c$ complemento, isto \'e, todos os elementos $\omega$ de $\Omega$ exceto aqueles que est\~{a}o em A.
		\item (ii) $A_1 \cup A_2 \cup A_3, \dots ,\cup A_n$  ou $ \bigcup_{j = 1}^{n} A_n $, \'{e} a
		uni\~{a}o, s\~{a}o todos os pontos $\omega \in \Omega$, que pertecem a pelo menos um $A_i$ com $ i =
		1,2, \dots,n$. 
		\item (iii) $A_1 \cap A_2 \cap A_3, \dots ,\cap A_n$  ou $ \bigcap_{j = 1}^{n} A_n $,
		\'{e} a intersec\c{c}\~{a}o, s\~{a}o todos os pontos $\omega \in \Omega$, pertencentes $A_i$ qualquer
		que seja o $i = 1, \dots, n$.
		\item (iv) $A - B$ ou $A \cup B^c$ \'{e} a diferen\c{c}a entre A e B, ou
		seja, s\~{a}o todos os elementos $ \omega \in \Omega$ pertencentes \`{a} A que n\~{a}o pertecem a B.
		\item (v) $A \bigtriangleup B$ \'{e} a diferen\c{c}a sim\'{e}trica entre A e B, todos os $\omega \in \Omega$
	pertecentes a $A \cup B$ exceto os $\omega \in \Omega$ que est\~{a}o na $A \cap B$. 
	\end{description}
As nota\c c\~oes de opera\c c\~oes permitem estabeler uma s\'erie de an\'alises dos eventos de interesse do
espa\c co amostral $\Omega$.  Considerando-se dois conjuntos A e B s\~{a}o ditos disjuntos ou mutuamente
exclusivos se, somente se, a interse\c{c}\~{a}o entre A e B \'{e} o conjunto vazio, ou de outra forma,
	\begin{align}
		\label {dis} \textrm{A e B disjuntos} \Leftrightarrow A \cap B = \emptyset. 
	\end{align}
Nota-se que a rela\c c\~ao estabelecida em \ref{dis} \'e uma forma de descrever os eventos do espa\c co
amostral $\Omega$ que n\~ao possuem nenhum elemento $\omega$ em comum.  Pode-se particionar o conjunto
$\Omega$ por subconjuntos desde que os subconjuntos escolhidos de $\Omega$ respeitem os dois crit\'erios a
seguir, 
	\begin{align*}
		\bigcup A_i = \Omega \qquad \textit{e} \qquad A_i \cap A_j = \emptyset, i \neq j.
	\end{align*}
Uma vez que os principais conceitos da linguagem dos conjuntos foram estabelecidos pode-se
considerar a defini\c c\~ao cl\'assica de probabilidade \cite{james}. 
	\begin{defin} 
		\label{class}
		Define-se a probabilidade cl\'assica de um evento $A$ ocorre pela raz\~ao entre o n\'umeros de casos
		favor\'aveis pelo n\'umero de casos poss\'iveis, ou seja, 
		\begin{align}
			P(A) = \frac{n(A)}{n(\Omega)}.
		\end{align} 
	\end{defin}
A rela\c c\~ao estabelecida em \ref{class} \'e v\'alida para um $\Omega$
finito. Al\'em disso, os eventos de $\Omega$ devem ser equiprov\'{a}veis, isto \'e, cada evento tem a
mesma probabilidade de ocorr\^{e}ncia.  A defini\c c\~ao \autoref{class} segue o princ\'ipio da
indifer\^encia, ou seja, a ocorr\^{e}ncia de um evento A n\~{a}o modificar a probabilidade de
ocorr\^{e}ncia de um evento B \cite{james}.  Para o lan\c{c}amento do dado tem-se, 
	\begin{equation*}
		P(i)= \frac{1}{6},\forall \quad i \in \Omega, \quad i = 1,2,3,4,5,6.
	\end{equation*} 
Portanto, independente da jogada e considerando-se que o dado n\~{a}o \'{e} viciado h\'a uma mesma probabilidade
associada independente do evento no espa\c{c}o amostral. Outro exemplo cl\'{a}ssico e o lan\c{c}amento de
uma moeda.  Descrevendo os eventos poss\'{i}veis por, 
	\begin{align*}
		\Omega = \{ \textrm{cara},
		\textrm{coroa} \}.  
	\end{align*}
Nota-se ambos os eventos s\~ao equiprov\'{a}veis, ou seja,
	\begin{equation*}
		P(\textrm{cara}) = P(\textrm{coroa}) = \frac{1}{2}.  
	\end{equation*} 
	\par
Uma outra defini\c c\~ao usual de probabilidade \'e a frequentista ou est\'{a}tistica.  
	\begin{defin}
		\label{freq}
		Dado um evento A definindo-se $n(A)$ o n\'umero de ocorr\^encias independentes do evento A.  Seja
		$n(\Omega)$ todos os casos poss\'{i}veis. Define-se a probabilidade frequentista como, 
	\end{defin}
	\begin{align*}
		P(A) = \displaystyle\lim_{n \to \infty} \frac{n (A)} {n(\Omega)}.  
	\end{align*} A
	express\~ao em \autoref{freq} \'e a fr\^{e}ncia da ocorr\^{e}cia do evento A para o \textit{n}
	sucientemente grande.  As defini\c{c}\~{o}es de probabilidade mencionadas no at\'e o presente momento
	utilizam-se um forte apelo a intui\c c\~ao. Com intuito de uma defini\c c\~ao mais rigorosa de
	probabilidade o matem\'{a}tico Kolmogorov estabeleceu um conjunto de axiomas \cite{magalhaes}. A seguir
	define-se uma probabilidade de forma geral.  
	\begin{defin}[Probabilidade] \label{kol} Uma fun\c{c}\~{a}o P
definida na $\sigma$-\'{a}lgebra F de subconjunto de $\Omega$ para valores no intervalo [0,1], \'{e} uma
probabilidade se satisfaz os Axiomas de Kolmogorov: \\ 
		\begin{enumerate} 
			\item $P(\Omega)= 1$; 
			\item Para todo	subconjunto $ A \in F, P(A) \geq 0$; 
			\item Qualquer que seja a sequ\^{e}ncia 
			$A_1, A_2, \dots \in F$
			mutuamente exclusivos tem-se, $$P( \bigcup\limits_{i = 1}^{\infty}A_i) = \sum\limits_{i =
			1}^{\infty} P(A_i).$$ 
		\end{enumerate}
	\end{defin}
O axioma (1) indica que uma probabilidade n\~ao
assume valores maiores que 1. Al\'em disso, pelo axioma(2) qualquer que seja a probabilidade essa
\'e n\~ao negativa. Os axiomas (1) e (2) garatem que uma probabilidade qualquer pertence ao
intervalo $[0,1]$. Finalmente, o axioma (3) permite  a decomposi\c c\~ao da probabilidade da
uni\~ao para eventos mutuamente exclusivos, ou seja, eventos independentes.  A trinca ($\Omega$,
F, P) \'e denominado espa\c{c}o da probabilidade os subconjuntos em F s\~{a}o os eventos, somente
a estes tem-se uma probabilidade associada. A defini\c c\~ao de probabilidade em \ref{kol} permite
a caracteriza\c c\~ao de propriedades com os eventos de $\Omega$. Primeiramente, considere o
espa\c{c}o de probabilidade $(\Omega, F, P )$ onde os conjuntos mencionados est\~{a}o contidos
neste espa\c{c}o tem-se: 
\begin{enumerate} 
	\item $P(A) = 1 - P(A^c)$
	\item Sendo A e B dois eventos
	quaisquer tem-se: \\ $P(B)= P(B \cap A) + P(B \cap A^c) $ \item Se $ A \subset $ B, ent\~{a}o
	$P(A) \leq P(B)$
	\item Regra da adi\c{c}\~{a}o de probabilidade  \\ $ P(A \cup B)= P(A) +
	P(B) - P(A \cap B)$ \item Para eventos quaisquer $A_1, A_2, \dots $ \\
	$$P(\bigcup\limits_{i = 1}^{\infty})A_i \leq \sum\limits_{i = 1}^{\infty} P(A_i).$$
	\end{enumerate}
Para exemplificar as propriedades apresentadas seja um  dado equilibrado esse
\'{e} lan\c{c}ado duas vezes e as faces resultantes observadas \cite{magalhaes}.
Considerando-se os seguintes eventos de interesse:

  \vspace{0.5cm} 
  \hspace{4cm}
  \begin{minipage}{9cm} 
	  A : a soma dos resultados \'{e} \'{i}mpar. \\ 
	  B : o resultado do primeiro lan\c{c}amento \'{i}mpar.  \\ C : o produto do resultado \'{e} impar.
  \end{minipage} 
  \vspace{0.5cm}
  \par 
Primeiramente deve-se definir um espa\c{c}o adequado para o experimento,
considerando-se $\Omega = \{1,2,3,4,5,6 \} \hspace{2mm} \text{x} \hspace{2mm} \{1,2,3,4,5,6 \}$. O conjunto
$\Omega$ \'e o produto cartesino de todos os resultados poss\'{i}veis, ou seja, todo ponto $\omega \in
\Omega$ pode ser descrito como $\omega = (\omega_1, \omega_2)$.  Utilizando-se como $\sigma$-\'{a}lgebra  o
conjunto das partes de $\Omega$ e P a probabilidade associada a cada ponto para cada um dos eventos em
$\Omega$ tem-se uma probabilidade uniforme, ou seja, $P(\{\omega \}) = \frac {1}{36}$. De fato, o resultado
\'{e} uma implica\c c\~ao do princ\'ipio fundamental da contagem.  Se uma tarefa tem k etapas e cada etapa
\textit{i} tem \textit{n} maneiras diferentes de ser realizada, ent\~{a}o  o n\'{u}mero total de
alternativas para realizar a tarefa \'{e} o produto $n_1 n_2 \dots n_k$ \cite{magalhaes}.  Para cada jogada
de um dos dado existe 6 possibilidades para o resultado.  Portanto, para o lan\c{c}amento de dois dados
tem-se 36 possibilidades significando que o espa\c{c}o amostral possui 36 pontos. A jogada do dado $D_1$
n\~{a}o interfere na jogada do dado $D_2$, ou seja, existe indep\^{e}ndencia de eventos.  Logo, o evento A
pode ser descrito como, 
	\begin{equation*} 
		A = \{ \omega = (\omega_1,\omega_2) \in \Omega; \omega_1 +
		\omega_2\ \ \textrm{\'{e} \'{i}mpar} \} 
	\end{equation*} ou de maneira equivalente, 
	\begin{equation*}
		  A = \{ \omega = (\omega_1,\omega_2) \in \Omega; \omega_1 + \omega_2  = 2n + 1, n \in \mathbb{N} \}.
	\end{equation*} 
Para o c\'{a}lculo da probabilidade do evento poderia-se elencar todos as poss\'{i}veis
combina\c{c}\~{o}es para este evento, ou seja, os pontos procurados s\~{a}o $\{1,2 \},\{1,4 \},\{1,6 \},\{2,1 \},\\
\{2, 3\}, \{2,5 \} ,\{3,2 \}, \{3,4 \}, \{3,6 \}, \{4,1 \}, \{4,3 \}, \{4,5 \}, \{5,2\},\{5,4 \},  
\{5,6 \}, \{6,1 \}, \{6, 3 \},\\ \{6, 5 \}$.  Verifica-se que o total \'e de 18 pontos
pontos a probabilidade associdada \'{e} $P(A) = \frac{18} {36} = \frac{1} {2}$.  De forma an\'aloga o
evento B pode ser descrito da seguinte forma, 
	\begin{equation*} 
		B = \{\omega = (\omega_1, \omega_2) \in
		\Omega; \omega_1 \textrm {\'{e} \'{i}mpar} \}. 
	\end{equation*}
Os pontos  s\~{a}o $\{1,1\},\{1,2 \},\{1,3 \}, \{1,4 \}, \{1, 5\}, \{1,6 \},\{3,1 \},\{3,2 \},\{3,3 \}, \{3,4 \}$,\\
$\{3,5\},\{3,6 \}, \{5,1 \},\{5,2 \},\{5,3 \}, \{5,4 \}, \{5, 5\}, \{5,6 \}$.  A probabilidade de
ocorr\^{e}ncia do evento B \'{e} $P(B) = \frac{18} {36} = \frac {1} {2}$. Finalmente, a
probabilidade associdada a C \'e $P(C) = \frac{9}{36} = \frac {1} {4}$. Com base nas
informa\c{c}\~{o}es das probabilidade do eventos A, B e C pode-se fazer conclus\~{o}es
 interessantes, por exemplo, percebe-se que $P(A \cap B) = \frac{1}{4}$. De fato, $A \cap B = \{1,1
\},\{1,3 \},\{1,5 \}, \{3,1 \}, \{3,3 \},\{3,5 \}, \{5,1 \}, \{5,3 \},\{5, 5 \}$, segue que,
		  \begin{eqnarray*}
			  P(A \cup B) &=&  P(A) + P(B) -P(A \cap B) \\ \nonumber P(A \cup B) &=&  \frac{1}
			  {2} + \frac{1} {2} - \frac{1} {4} = \frac{3} {4}. 
		  \end{eqnarray*}
Consequentemente, a probabilidade da soma ser \'{i}mpar e o resultado do primeiro lan\c{c}amento ser \'{i}mpar
corresponde a $\frac{1} {4}$. E a probabilidade da ocorr\^{e}ncia da soma dos resultados ser
\'{i}mpar ou resultado do primeiro lan\c{c}amento ser \'{i}mpar corresponde a $\frac{3} {4}$.
Geralmente em problemas de natureza pr\'atica h\'a a necessidade do c\'alculo da probabilidade um
certo A dado que o um evento b ocorreu.  Por exemplo, perguntas do tipo "Qual a probabilidade que
ocorra uma chuva dado que a temperatura diminuiu". Esse tipo de pergunta onde o interesse de um
evento depende da ocorr\^encia de outro evento fundamenta a no\c c\~ao de probabilidade condicional.
	\begin{defin}[Probabilidade Condicional]
		\label{dcon}
		Considerando-se os eventos A e B em ($\Omega$,
		F, P) onde $P(B)  > 0$. A probabilidade de ocorr\^{e}ncia do evento A tal que B ocorreu \'e dado
		por,
		\begin{equation*}
			P(A|B) = \frac{P(A \cap B)} {P(B)} 
		\end{equation*} para o caso $P(B) = 0$
		define-se $P(A|B ) = P(A) $.
	\end{defin}
A defini\c c\~ao de probabilidade condicional possue
caracter\'isticas interessantes, primeiramente o fato da $P(A \cap B) = P(A)$ quando $P(B) = 0$
permite que a probabilidade condicional tenha semelhan\c cas com a nota\c c\~ao usual de uma fun\c
c\~ao uma vez que a probabilidade condicional $P(A \cap B)$ depender\'a somente de $A$ \cite{james}.
Outro fator de relev\^ancia deve-se ao interesse te\'orico uma vez que probabilidade condicional
permitir a decomposi\c{c}\~{a}o de probalidades que possuem uma dif\'{i}cil caracteriza\c{c}\~{a}o
por meio de probabilidade condicionais mais simples\cite{magalhaes}. Por fim, baseando-se na
probabilidade condicional \'e poss\'ivel o produto de probabilidades.  
	\begin{prop}[Regra do produto de probabilidades \cite{magalhaes}] 
		Para os eventos $A_1,A_2, \dots, A_n$ em ($\Omega$, F, P)
		com $P( \bigcap\limits_{i = 1}^{\infty}) > 0 $.  O produto das probabilidades \'{e} dado por,
		\begin{equation*} P(A_1 \cap A_2, \cap A_3 \dots \cap A_n) = P(A_1)P(A_2 | A_1) \dots P(A_n|A_1
			\cap A_2 \dots A_{n-1}).  
		\end{equation*}
	\end{prop}
Uma aplica\c{c}\~{a}o direta da regra
do produto de probabilidades \'{e} a lei da probabilidade total.  
	\begin{teo}[Lei da Probabilidade Total\cite{magalhaes}]
		Supondo-se que os eventos $C_1,C_2,\dots, C_n$ em ($\Omega$, F, P) formam
		uma parti\c{c}\~{a}o de $\Omega$ e que para qualquer $C_n$ tem-se $C_n > 0$.  Ent\~{a}o, para
		qualquer evento A neste espa\c{c}o de probabilidade. A probabilidade do evento A \'{e} dada por,
			\begin{equation*}
				P(A) = \sum\limits_{i = 1}^{n} P(C_i)P(A|C_i).  
			\end{equation*} 
	\end{teo} 
O membro do direito lei de probabilidade total \'{e} formada produtos envolvento as probabilidades
condicionais.  Por meio da probabilidade condicional pode-se caracterizar de outra forma a
independ\^encia de eventos.  
	\begin{defin}[Independ\^{e}ncia de dois eventos\cite{magalhaes}]
		\label{ind}
		Dados dois eventos A e B em ($\Omega$, F, P) s\~{a}o ditos independentes quando a ocorr\^{e}ncia do
		evento A n\~{a}o influ\^{e}ncia na ocorr\^{e}ncia do evento do evento B, isto \'{e},
			\begin{align}
				\label{ind1}
				P(A|B) = P(A). 
			\end{align}
		Conforme a defini\c{c}\~{a}o \autoref{dcon} de probabilidade condicional, 
		\begin{align}
			\label{ind2} 
			P(A|B) = \frac{P(A \cap B)} {P(B)} 
		\end{align}
		de \ref{ind1} e \ref{ind2} obt\^{e}m-se uma representa\c{c}\~{a}o
		equivalente para a independ\^{e}ncia de eventos dada por, 
		\begin{equation*}
			P(A \cap B) = P(A)P(B).  
		\end{equation*}  
	\end{defin}
Portanto, a independ\^{e}ncia para dois
eventos pode ser analisada utilizando-se o c\'{a}lculo da probabiliadade associada com a
intersec\c{c}\~{a}o facilitando-se a an\'{a}lise dos eventos e futuras conclus\~{o}es sobre o
fen\^{o}meno observado. Com base na defini\c c\~ao \autoref{ind} \'e definida a indep\^{e}ndencia de v\'{a}rios eventos.
	\begin{defin} 
		\label{indg}
		Os eventos $A_1,A_2,\dots ,A_n $ em ($\Omega$, F, P) s\~{a}o
		independentes se para toda cole\c{c}\~{a}o de \'{i}ndices $ 1 \leq i_1 \leq i_2 < \dots i_k \leq
		n $ \'e verdadeiro que 
		\begin{equation*}
			P(A_{i_1} \cap A_{i_2} \cap A_{i_3} \dots \cap A_{i_k}) = P(A_{i_1})P(A_{i_2}) \dots P(A_{i_k}).
		\end{equation*} 
	\end{defin}
Consequentemente por \ref{ind} e \ref{indg}, \'{e} poss\'{i}vel analisar a independ\^{e}ncia de eventos com base no
c\'{a}lculo de probabilidades. Outro fator importante deve-se que pelo conhecimento da indep\^encia eventos 
facilitar a an\'alise do evento de interesse, por exemplo, uma moeda \'{e} lan\c{c}ada duas
vezes. Sejam os eventos:

   \vspace{0.5cm}
   \hspace{4cm}
   \begin{minipage}{9cm} 
	   A : Sair cara; \\ B : Sair coroa. \\ 
   \end{minipage}
Nota-se que os eventos A e B s\~{a}o independentes. De fato, a ocorr\^{e}ncia do evento A ou B no primeiro
lan\c{c}amento em nada influ\^{e}ncia o resultado para o segundo lan\c{c}amento. A probabilidade de sair
 cara no segundo lan\c{c}amento dado que saiu coroa no primeiro lan\c{c}amento corresponde a $\frac {1}
   {4}$.  Pois, como os eventos s\~{a}o indenpedentes, 
   \begin{eqnarray*}
	   P(B|A) &=& P(B) \\ \nonumber P(B|A)
	   &=&  \frac{1} {2}. \nonumber 
   \end{eqnarray*} \'{e} de imediato que, 
   \begin{eqnarray*} P(B \cap A ) &=&
	   P(B)P(A) \\ P(B \cap A) &=&  \left( \frac{1} {2}  \right) \left( \frac{1} {2}  \right) = \frac{1} {4}
   \end{eqnarray*} 
O mesmo resultado poderia ser obtido c\'{a}lculando-se a probabilidade pela forma
cl\'{a}ssica, ou seja elecando-se os resultados poss\'iveis, (cara, cara), (cara, coroa), (coroa, cara),
(coroa, coroa).  O evento de interesse \'{e} (coroa, cara) para o experimento $\Omega = 4 $ tem-se de
imediato que, 
\begin{equation*}
	P( \textrm{coroa, cara}) = \frac {1} {4}. 
\end{equation*} A primeira forma
e a segunda tiveram o mesmo resultado, contudo, se o evento em um espa\c{c}o amostral \'e constitu\'ido por
quantidade de elementos suficientemente grande a primeira forma \'{e} mais atraente. No estudo de
fen\^{o}menos alet\'{o}rios \'{e} comum o interesse em quantidades que possam ser associadas ao evento.
Antes da realiza\c{c}\~{a}o de qualquer experimento ou ocorr\^{e}ncia de qualquer fen\^{o}meno de natureza
aleat\'{o}ria n\~{a}o \'{e} comum ter o resultado. Contudo, uma vez que o espa\c{c}o de probabilidade
esteja bem definido \'{e} poss\'{i}vel observar qualquer evento de interesse na $\sigma$-\'{a}lgebra. Dessa
forma, tamb\'{e}m \'{e} poss\'{i}vel atribuir probabilidade para as fun\c{c}\~{o}es que descrevem o evento.
   Nessa perspectiva fundamenta-se o conceito do termo vari\'{a}vel aleat\'{o}ria.  
	\begin{defin}[Vari\'{a}vel
	aleat\'{o}ria] Dado um espa\c{c}o de probabilidade ($\Omega$, F, P), define-se uma vari\'{a}vel
	aleat\'{o}ria como uma fun\c{c}\~{a}o X: $\Omega \rightarrow \mathbb{R}$ tal que 
		\begin{equation*}
		 	\label{v.a}
		  	X^{-1}(I) = \{ \omega \in \Omega; X(\omega) \in I \} \in F.  
		\end{equation*} 
	qualquer que seja o conjunto $I \subset \mathbb{R}$ pertecente a $\sigma$-\'{a}lgebra.  
	\end{defin}
Pela defini\c{c}\~{a}o \ref{v.a} existe uma fun\c{c}\~{a}o de $\Omega$ em $\mathbb{R}$. Pela qual \'e poss\'ivel
ser estabelecer sua inversa de $\Omega$ em $\mathbb{R}$. A exig\^{e}ncia que a inversa pertence a
$\sigma$-\'{a}lgebra \'{e} importante, pois \'{e} poss\'{i}vel garantir as opera\c{c}\~{o}es com as
probabilidades para os elementos que estejam na $\sigma$-\'{a}lgebra. Considerando-se $\Omega = \{1,2,3,4
\}$ e seja $ F = \{ \emptyset, \Omega, \{1,2 \}, \{3,4 \} \} $  definindo-se os conjuntos $A = \{1,2 \}$ e
$B = \{1,3 \}$ deve-se verificar se \'{e} poss\'{i}vel $I_A$ e $I_B$ sejam vari\'{a}veis
aleat\'{o}rias\cite{magalhaes}.  Primeiramente, definindo-se $X^{-1}$ por 
	\begin{equation*}
		( \omega \in\Omega; I_{A} \in (-\infty, x]) = \left\{
			\begin{array}{lcl} \emptyset, & \mbox {se} &  x < 0, \\ A^{c},
	   		& \mbox {se} &  0 \leq x < 1, \\ \Omega,    & \mbox {se} & x \geq 1 
   			\end{array} \right.
	\end{equation*} nota-se que para qualquer intervalo da forma  $ (-\infty, x ] \in F $ uma vez que
ambos os subconjuntos $\emptyset, A^{c}, \Omega \in F$ est\~ao em F, dessa forma $I_A$ \'{e}
vari\'{a}vel aleat\'{o}ria.  Contudo, utilizando-se a mesma defini\c{c}\~{a}o de $I_A$ em $I_B$ ocorrer
que $B^{c}$ n\~{a}o pertence a F. De fato, $B^{c} = \{2,4 \}\ \not \in F$ nessas condi\c c\~oes $I_B$
n\~{a}o \'{e} uma  vari\'{a}vel aleat\'{o}ria.  Por outro lado, poderia-se definir uma outra
$\sigma$-\'{a}lgebra que comporta-se o $I_B$ permitindo-se que $I_B$ seja uma vari\'{a}vel
aleat\'{o}ria. A formula\c{c}\~{a}o de uma vari\'{a}vel aleat\'oria est\'{a} sempre associada
diretamente  com a $\sigma$-\'{a}lgebra. Os eventos da forma $( - \infty, x)$ possuem um interesse em
particular, pois, constituem uma importante fun\c c\~ao no estudo da probabilidade denominada
fun\c{c}\~{a}o de distribui\c{c}\~{a}o de probabilidade. No intuito de facilitar o entendimento quando
houve men\c{c}\~{a}o a $\omega$ em algum evento da forma $ \{ \omega \in \Omega; X(\omega) \in I)\} $
este ser\'{a} omitido, portanto a representa\c c\~ao do evento de interesse ser\'a $ [ X \in I ]$.
Desta forma, os eventos como $I = (-\infty, x]$ seram representados por $[X < 0]$ em vez de $P(\{\omega \in
\Omega: X(\omega) \in I \})$ ser\'{a} adotada a nomenclatura $P(X \in I)$.  Por este tipo de
simplica\c{c}\~{a}o a fun\c{c}\~{a}o de distribui\c{c}\~{a}o de probabilidade tamb\'{e}m \'{e}
conhecida por fun\c{c}\~{a}o de probabilidade acumulada pelo acumulo de probabilidade no intervalo
real\cite{magalhaes}.

  \begin{defin}[Fun\c{c}\~{a}o de distribui\c{c}\~{a}o de probabilidade]
	  Seja X uma vari\'{a}vel aleat\'{o}ria
	  em um espa\c{c}o de probabilidade ($\Omega$, F, P). A fun\c{c}\~{a}o de distribui\c{c}\~{a}o \'{e} dada por,
  	\begin{equation*} 
		F_{x} (x) = P( X \in (-\infty, x]) = P(X \leq x). 
	\end{equation*} \label{fdp} 
  \end{defin}
O conhecimento da defini\c c\~ao \ref{fdp} permite obter qualquer informa\c{c}\~{a}o sobre a vari\'{a}vel a ser examinada.
Apesar da vari\'{a}vel aleat\'oria assumir valores num subconjunto dos reais a sua fun\c{c}\~{a}o de
distribui\c{c}\~{a}o assumir valores em todos os reais \cite{magalhaes}. Al\'{e}m da fun\c{c}\~{a}o de
distribui\c{c}\~{a}o existe uma outra fun\c{c}\~{a}o de import\^ancia semelhante para o estudo da
probabilidade conhecida como fun\c{c}\~{a}o de probabilidade.  Contudo, deve ser ressaltado que uma
vari\'{a}vel aleat\'{o}ria X \'{e} classificada como: discreta ou cont\'{i}nua. Uma vari\'{a}vel \'{e}
classificada como discreta quando assumir somente um n\'{u}mero enumer\'{a}vel de valores finito ou infinito.
Desta forma, uma fun\c c\~ao de probabilidade para esse tipo de vari\'avel aleat\'oria 
atribuir valor a cada um dos poss\'{i}veis valores assumidos pela vari\'{a}vel
aleat\'{o}ria, isto \'e, tomando-se X como uma vari\'{a}vel aleat\'{o}ria que assumir valores em $x_1, x_2,
\dots $, para $i = 1,2, \dots,$ portanto, a fun\c c\~ao de probabilidade \'e dada por, 
	\begin{equation*}
		P(x_i) = P(X = x_i) = P(\omega \in \Omega; X(\omega) = x). 
	\end{equation*} Em rela\c c\~ao ao caso
discreto a fun\c{c}\~{a}o de probabilidade deve satisfazer duas propriedades, 
	\begin{description}
		\centering
	  	\item $p_1$: $ 0 \leq p(x_i) \leq 1, \forall i = 1,2, \dots $, 
		\item $p_2$: $ \sum\limits_{i}^{} p(x_i)= 1. \qquad \qquad \quad \ \ \ \ $ 
  	\end{description} 
As propriedades $p_1$ e $p_2$ confirmam os
axiomas de probabilidade.  De fato, por $p_1$ a probabilidade somente assumir\'{a} valores n\~{a}o negativos
no intervalo $[0,1]$ e por $p_2$ o somat\'{o}rio de todas as probabilidades asssocidas ao evento dever ser
igual a 1.  Para o caso cont\'{i}nuo a ideia \'e an\'aloga, contudo como a vari\'avel aleat\'oria \'e cont\'inua a
utiliza\c{c}\~{a}o de uma integral ao inv\'{e}s do somat\'{o}rio \'{e} o ideal, portanto, uma vari\'{a}vel
aleat\'{o}ria X ser\'{a} classificada como cont\'{i}nua como quando exite uma fun\c{c}\~{a}o $f$ n\~{a}o
negativa que satisfa\c{c}a, 
	\begin{description} \centering 
		\item $p_1:$ $\int_{-\infty}^{\infty} f(\omega)
		 d\omega \geq 0 , \forall x \in \mathbb{R} $, 
		\item $p_2:$ $\int_{-\infty}^{\infty} f(\omega) d\omega =
		1.\qquad \quad \ \ $ 
	\end{description} 
Observando-se que a constru\c{c}\~{a}o de tais
fun\c{c}\~{o}es dever existir uma espa\c{c}o de probabilidade ($\Omega$, F, P) bem definido.  Pelas fun\c{c}\~{o}es
de ditribui\c{c}\~{a}o e de probabilidade consegue-se a maioria das informa\c{c}\~{o}es sobre o evento.
Por fim, com a fundamenta\c c\~ao adquirida pode-se abordar um dos principais conceitos da teoria da
probabilidade o de valor esperado, esperan\c{c}a matem\'{a}tica, ou m\'{e}dia de
uma vari\'{a}vel aleat\'oria. A no\c c\~ao de valor esperado \'e extremamente \'{u}til na an\'{a}lise de fen\^{o}menos
aleat\'{o}rios justamente por
ser comportar como a m\'{e}dia permitindo informa\c{c}\~{o}es do problema do problema observado..
Para o caso discreto \'e definido como, 
	\begin{align}
		\label{dise}
		E(X) = \sum\limits_{i = 1}^{\infty}x_ip_X(x_i).
	\end{align}
Nota-se que o valor esperado \'e
definido como um somat\'orio na rela\c c\~ao \ref{dise}, pois, a vari\'avel aleat\'oria X \'e discreta, ou seja, assume
valores $x_i$ para $i= 1,2, \dots$, por exemplo,
considerando-se o lan\c camento de um dado onde a vari\'avel X representar a face do dado. Desta forma, X \'e uma
vari\'avel discreta
e assume valores no conjunto
$\Omega = \{1,2,3,4,5,6\}$. O Valor esperado de X \'e dado por,
	\begin{align*}
		E(X) = \frac{1}{6}(1 + 2 + 3 + 4 + 5 + 6) = \frac{21}{6} = \frac{7}{2}
	\end{align*}
desde que soma esteja bem determinada e o espa\c{c}o de probabilidade bem definido.
Nota-se que o valor de X nesse caso n\~ao pertence ao conjunto $\Omega$, isto \'e, o valor esperado indica qual o valor
que se espera para a repeti\c c\~ao do evento um n\'umero  $n$ vezes, sendo $n$ um n\'umero grande. Portanto, como todos
o resultados do evento descrito s\~ao equiprov\'aveis, logo, pode-se concluir que para a repeti\c c\~ao um n\'umero
suficiente grande de vezes a m\'edia aritm\'etica tende a $\frac{7}{2}$. Para o caso
cont\'{i}nuo, a ideia \'{e} an\'{a}loga,
	\begin{align*}
		E(X) = \int\limits_{-\infty}^{\infty} xf(x) dx, 
	\end{align*}
desde que a integral esteja bem definida, assim como o espa\c{c}o de probabilidade. 
Nesse cap\'itulo foram abordados os principais conceitos de probabilidade relevantes para pesquisa. Primeiramente, a
constru\c c\~ao da ideia de probabilidade e as rela\c c\~oes que fundamentam seu uso. Al\'em da estrutura\c c\~ao dos 
conceitos como o valor de esperado de fundamental import\^ancia para o despacho de energia no c\'alculo do custo
esperado. 
